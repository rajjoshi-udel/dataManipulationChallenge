% Options for packages loaded elsewhere
% Options for packages loaded elsewhere
\PassOptionsToPackage{unicode}{hyperref}
\PassOptionsToPackage{hyphens}{url}
\PassOptionsToPackage{dvipsnames,svgnames,x11names}{xcolor}
%
\documentclass[
  letterpaper,
  DIV=11,
  numbers=noendperiod]{scrartcl}
\usepackage{xcolor}
\usepackage{amsmath,amssymb}
\setcounter{secnumdepth}{-\maxdimen} % remove section numbering
\usepackage{iftex}
\ifPDFTeX
  \usepackage[T1]{fontenc}
  \usepackage[utf8]{inputenc}
  \usepackage{textcomp} % provide euro and other symbols
\else % if luatex or xetex
  \usepackage{unicode-math} % this also loads fontspec
  \defaultfontfeatures{Scale=MatchLowercase}
  \defaultfontfeatures[\rmfamily]{Ligatures=TeX,Scale=1}
\fi
\usepackage{lmodern}
\ifPDFTeX\else
  % xetex/luatex font selection
\fi
% Use upquote if available, for straight quotes in verbatim environments
\IfFileExists{upquote.sty}{\usepackage{upquote}}{}
\IfFileExists{microtype.sty}{% use microtype if available
  \usepackage[]{microtype}
  \UseMicrotypeSet[protrusion]{basicmath} % disable protrusion for tt fonts
}{}
\makeatletter
\@ifundefined{KOMAClassName}{% if non-KOMA class
  \IfFileExists{parskip.sty}{%
    \usepackage{parskip}
  }{% else
    \setlength{\parindent}{0pt}
    \setlength{\parskip}{6pt plus 2pt minus 1pt}}
}{% if KOMA class
  \KOMAoptions{parskip=half}}
\makeatother
% Make \paragraph and \subparagraph free-standing
\makeatletter
\ifx\paragraph\undefined\else
  \let\oldparagraph\paragraph
  \renewcommand{\paragraph}{
    \@ifstar
      \xxxParagraphStar
      \xxxParagraphNoStar
  }
  \newcommand{\xxxParagraphStar}[1]{\oldparagraph*{#1}\mbox{}}
  \newcommand{\xxxParagraphNoStar}[1]{\oldparagraph{#1}\mbox{}}
\fi
\ifx\subparagraph\undefined\else
  \let\oldsubparagraph\subparagraph
  \renewcommand{\subparagraph}{
    \@ifstar
      \xxxSubParagraphStar
      \xxxSubParagraphNoStar
  }
  \newcommand{\xxxSubParagraphStar}[1]{\oldsubparagraph*{#1}\mbox{}}
  \newcommand{\xxxSubParagraphNoStar}[1]{\oldsubparagraph{#1}\mbox{}}
\fi
\makeatother

\usepackage{color}
\usepackage{fancyvrb}
\newcommand{\VerbBar}{|}
\newcommand{\VERB}{\Verb[commandchars=\\\{\}]}
\DefineVerbatimEnvironment{Highlighting}{Verbatim}{commandchars=\\\{\}}
% Add ',fontsize=\small' for more characters per line
\usepackage{framed}
\definecolor{shadecolor}{RGB}{241,243,245}
\newenvironment{Shaded}{\begin{snugshade}}{\end{snugshade}}
\newcommand{\AlertTok}[1]{\textcolor[rgb]{0.68,0.00,0.00}{#1}}
\newcommand{\AnnotationTok}[1]{\textcolor[rgb]{0.37,0.37,0.37}{#1}}
\newcommand{\AttributeTok}[1]{\textcolor[rgb]{0.40,0.45,0.13}{#1}}
\newcommand{\BaseNTok}[1]{\textcolor[rgb]{0.68,0.00,0.00}{#1}}
\newcommand{\BuiltInTok}[1]{\textcolor[rgb]{0.00,0.23,0.31}{#1}}
\newcommand{\CharTok}[1]{\textcolor[rgb]{0.13,0.47,0.30}{#1}}
\newcommand{\CommentTok}[1]{\textcolor[rgb]{0.37,0.37,0.37}{#1}}
\newcommand{\CommentVarTok}[1]{\textcolor[rgb]{0.37,0.37,0.37}{\textit{#1}}}
\newcommand{\ConstantTok}[1]{\textcolor[rgb]{0.56,0.35,0.01}{#1}}
\newcommand{\ControlFlowTok}[1]{\textcolor[rgb]{0.00,0.23,0.31}{\textbf{#1}}}
\newcommand{\DataTypeTok}[1]{\textcolor[rgb]{0.68,0.00,0.00}{#1}}
\newcommand{\DecValTok}[1]{\textcolor[rgb]{0.68,0.00,0.00}{#1}}
\newcommand{\DocumentationTok}[1]{\textcolor[rgb]{0.37,0.37,0.37}{\textit{#1}}}
\newcommand{\ErrorTok}[1]{\textcolor[rgb]{0.68,0.00,0.00}{#1}}
\newcommand{\ExtensionTok}[1]{\textcolor[rgb]{0.00,0.23,0.31}{#1}}
\newcommand{\FloatTok}[1]{\textcolor[rgb]{0.68,0.00,0.00}{#1}}
\newcommand{\FunctionTok}[1]{\textcolor[rgb]{0.28,0.35,0.67}{#1}}
\newcommand{\ImportTok}[1]{\textcolor[rgb]{0.00,0.46,0.62}{#1}}
\newcommand{\InformationTok}[1]{\textcolor[rgb]{0.37,0.37,0.37}{#1}}
\newcommand{\KeywordTok}[1]{\textcolor[rgb]{0.00,0.23,0.31}{\textbf{#1}}}
\newcommand{\NormalTok}[1]{\textcolor[rgb]{0.00,0.23,0.31}{#1}}
\newcommand{\OperatorTok}[1]{\textcolor[rgb]{0.37,0.37,0.37}{#1}}
\newcommand{\OtherTok}[1]{\textcolor[rgb]{0.00,0.23,0.31}{#1}}
\newcommand{\PreprocessorTok}[1]{\textcolor[rgb]{0.68,0.00,0.00}{#1}}
\newcommand{\RegionMarkerTok}[1]{\textcolor[rgb]{0.00,0.23,0.31}{#1}}
\newcommand{\SpecialCharTok}[1]{\textcolor[rgb]{0.37,0.37,0.37}{#1}}
\newcommand{\SpecialStringTok}[1]{\textcolor[rgb]{0.13,0.47,0.30}{#1}}
\newcommand{\StringTok}[1]{\textcolor[rgb]{0.13,0.47,0.30}{#1}}
\newcommand{\VariableTok}[1]{\textcolor[rgb]{0.07,0.07,0.07}{#1}}
\newcommand{\VerbatimStringTok}[1]{\textcolor[rgb]{0.13,0.47,0.30}{#1}}
\newcommand{\WarningTok}[1]{\textcolor[rgb]{0.37,0.37,0.37}{\textit{#1}}}

\usepackage{longtable,booktabs,array}
\usepackage{calc} % for calculating minipage widths
% Correct order of tables after \paragraph or \subparagraph
\usepackage{etoolbox}
\makeatletter
\patchcmd\longtable{\par}{\if@noskipsec\mbox{}\fi\par}{}{}
\makeatother
% Allow footnotes in longtable head/foot
\IfFileExists{footnotehyper.sty}{\usepackage{footnotehyper}}{\usepackage{footnote}}
\makesavenoteenv{longtable}
\usepackage{graphicx}
\makeatletter
\newsavebox\pandoc@box
\newcommand*\pandocbounded[1]{% scales image to fit in text height/width
  \sbox\pandoc@box{#1}%
  \Gscale@div\@tempa{\textheight}{\dimexpr\ht\pandoc@box+\dp\pandoc@box\relax}%
  \Gscale@div\@tempb{\linewidth}{\wd\pandoc@box}%
  \ifdim\@tempb\p@<\@tempa\p@\let\@tempa\@tempb\fi% select the smaller of both
  \ifdim\@tempa\p@<\p@\scalebox{\@tempa}{\usebox\pandoc@box}%
  \else\usebox{\pandoc@box}%
  \fi%
}
% Set default figure placement to htbp
\def\fps@figure{htbp}
\makeatother





\setlength{\emergencystretch}{3em} % prevent overfull lines

\providecommand{\tightlist}{%
  \setlength{\itemsep}{0pt}\setlength{\parskip}{0pt}}



 


\KOMAoption{captions}{tableheading}
\makeatletter
\@ifpackageloaded{tcolorbox}{}{\usepackage[skins,breakable]{tcolorbox}}
\@ifpackageloaded{fontawesome5}{}{\usepackage{fontawesome5}}
\definecolor{quarto-callout-color}{HTML}{909090}
\definecolor{quarto-callout-note-color}{HTML}{0758E5}
\definecolor{quarto-callout-important-color}{HTML}{CC1914}
\definecolor{quarto-callout-warning-color}{HTML}{EB9113}
\definecolor{quarto-callout-tip-color}{HTML}{00A047}
\definecolor{quarto-callout-caution-color}{HTML}{FC5300}
\definecolor{quarto-callout-color-frame}{HTML}{acacac}
\definecolor{quarto-callout-note-color-frame}{HTML}{4582ec}
\definecolor{quarto-callout-important-color-frame}{HTML}{d9534f}
\definecolor{quarto-callout-warning-color-frame}{HTML}{f0ad4e}
\definecolor{quarto-callout-tip-color-frame}{HTML}{02b875}
\definecolor{quarto-callout-caution-color-frame}{HTML}{fd7e14}
\makeatother
\makeatletter
\@ifpackageloaded{caption}{}{\usepackage{caption}}
\AtBeginDocument{%
\ifdefined\contentsname
  \renewcommand*\contentsname{Table of contents}
\else
  \newcommand\contentsname{Table of contents}
\fi
\ifdefined\listfigurename
  \renewcommand*\listfigurename{List of Figures}
\else
  \newcommand\listfigurename{List of Figures}
\fi
\ifdefined\listtablename
  \renewcommand*\listtablename{List of Tables}
\else
  \newcommand\listtablename{List of Tables}
\fi
\ifdefined\figurename
  \renewcommand*\figurename{Figure}
\else
  \newcommand\figurename{Figure}
\fi
\ifdefined\tablename
  \renewcommand*\tablename{Table}
\else
  \newcommand\tablename{Table}
\fi
}
\@ifpackageloaded{float}{}{\usepackage{float}}
\floatstyle{ruled}
\@ifundefined{c@chapter}{\newfloat{codelisting}{h}{lop}}{\newfloat{codelisting}{h}{lop}[chapter]}
\floatname{codelisting}{Listing}
\newcommand*\listoflistings{\listof{codelisting}{List of Listings}}
\makeatother
\makeatletter
\makeatother
\makeatletter
\@ifpackageloaded{caption}{}{\usepackage{caption}}
\@ifpackageloaded{subcaption}{}{\usepackage{subcaption}}
\makeatother
\usepackage{bookmark}
\IfFileExists{xurl.sty}{\usepackage{xurl}}{} % add URL line breaks if available
\urlstyle{same}
\hypersetup{
  pdftitle={Data Manipulation Challenge},
  colorlinks=true,
  linkcolor={blue},
  filecolor={Maroon},
  citecolor={Blue},
  urlcolor={Blue},
  pdfcreator={LaTeX via pandoc}}


\title{Data Manipulation Challenge}
\usepackage{etoolbox}
\makeatletter
\providecommand{\subtitle}[1]{% add subtitle to \maketitle
  \apptocmd{\@title}{\par {\large #1 \par}}{}{}
}
\makeatother
\subtitle{A Mental Model for Method Chaining in Pandas}
\author{}
\date{}
\begin{document}
\maketitle


\section{🔗 Data Manipulation Challenge - A Mental Model for Method
Chaining in
Pandas}\label{data-manipulation-challenge---a-mental-model-for-method-chaining-in-pandas}

\begin{tcolorbox}[enhanced jigsaw, breakable, left=2mm, opacitybacktitle=0.6, toptitle=1mm, toprule=.15mm, titlerule=0mm, opacityback=0, coltitle=black, leftrule=.75mm, arc=.35mm, colbacktitle=quarto-callout-important-color!10!white, rightrule=.15mm, title=\textcolor{quarto-callout-important-color}{\faExclamation}\hspace{0.5em}{📊 Challenge Requirements In Section
\hyperref[student-analysis-section]{Student Analysis Section}}, colback=white, bottomtitle=1mm, bottomrule=.15mm, colframe=quarto-callout-important-color-frame]

\begin{itemize}
\tightlist
\item
  Complete all discussion questions for the seven mental models (plus
  some extra requirements for higher grades)
\end{itemize}

\end{tcolorbox}

\begin{tcolorbox}[enhanced jigsaw, breakable, left=2mm, opacitybacktitle=0.6, toptitle=1mm, toprule=.15mm, titlerule=0mm, opacityback=0, coltitle=black, leftrule=.75mm, arc=.35mm, colbacktitle=quarto-callout-important-color!10!white, rightrule=.15mm, title=\textcolor{quarto-callout-important-color}{\faExclamation}\hspace{0.5em}{🎯 Note on Python Usage}, colback=white, bottomtitle=1mm, bottomrule=.15mm, colframe=quarto-callout-important-color-frame]

\textbf{Recommended Workflow: Use Your Existing Virtual Environment} If
you completed the Tech Setup Challenge Part 2, you already have a
virtual environment set up! Here's how to use it for this new challenge:

\begin{enumerate}
\def\labelenumi{\arabic{enumi}.}
\tightlist
\item
  \textbf{Clone this new challenge repository} (see Getting Started
  section below)
\item
  \textbf{Open the cloned repository in Cursor}
\item
  \textbf{Set this project to use your existing Python interpreter:}

  \begin{itemize}
  \tightlist
  \item
    Press \texttt{Ctrl+Shift+P} → ``Python: Select Interpreter''
  \item
    Navigate to and choose the interpreter from your existing virtual
    environment (e.g.,
    \texttt{your-previous-project/venv/Scripts/python.exe})
  \end{itemize}
\item
  \textbf{Activate the environment in your terminal:}

  \begin{itemize}
  \tightlist
  \item
    Open terminal in Cursor (`Ctrl + ``)
  \item
    Navigate to your previous project folder where you have the
    \texttt{venv} folder
  \item
    \textbf{💡 Pro tip:} You can quickly navigate by typing \texttt{cd}
    followed by dragging the folder from your file explorer into the
    terminal
  \item
    Activate using the appropriate command for your system:

    \begin{itemize}
    \tightlist
    \item
      \textbf{Windows Command Prompt:}
      \texttt{venv\textbackslash{}Scripts\textbackslash{}activate}
    \item
      \textbf{Windows PowerShell:}
      \texttt{.\textbackslash{}venv\textbackslash{}Scripts\textbackslash{}Activate.ps1}
    \item
      \textbf{Mac/Linux:} \texttt{source\ venv/bin/activate}
    \end{itemize}
  \item
    You should see \texttt{(venv)} at the beginning of your terminal
    prompt
  \end{itemize}
\item
  \textbf{Install additional packages if needed:}
  \texttt{pip\ install\ pandas\ numpy\ matplotlib\ seaborn}
\end{enumerate}

\begin{quote}
\textbf{⚠️ Cloud Storage Warning}

\textbf{Avoid using Google Drive, OneDrive, or other cloud storage for
Python projects!} These services can cause issues with: - Package
installations failing due to file locking - Virtual environment
corruption - Slow performance during pip operations

\textbf{Best practice:} Keep your Python projects in a local folder like
\texttt{C:\textbackslash{}Users\textbackslash{}YourName\textbackslash{}Documents\textbackslash{}}
or \texttt{\textasciitilde{}/Documents/} instead of cloud-synced
folders.
\end{quote}

\textbf{Alternative: Create a New Virtual Environment} If you prefer a
fresh environment, follow the Quarto documentation:
\url{https://quarto.org/docs/projects/virtual-environments.html}. Be
sure to follow the instructions to activate the environment, set it up
as your default Python interpreter for the project, and install the
necessary packages (e.g.~pandas) for this challenge. For installing the
packages, you can use the \texttt{pip\ install\ -r\ requirements.txt}
command since you already have the requirements.txt file in your
project. Some steps do take a bit of time, so be patient.

\textbf{Why This Works:} Virtual environments are portable - you can use
the same environment across multiple projects, and Cursor automatically
activates it when you select the interpreter!

\end{tcolorbox}

\subsection{The Problem: Mastering Data Manipulation Through Method
Chaining}\label{the-problem-mastering-data-manipulation-through-method-chaining}

\textbf{Core Question:} How can we efficiently manipulate datasets using
\texttt{pandas} method chaining to answer complex business questions?

\textbf{The Challenge:} Real-world data analysis requires combining
multiple data manipulation techniques in sequence. Rather than creating
intermediate variables at each step, method chaining allows us to write
clean, readable code that flows logically from one operation to the
next.

\textbf{Our Approach:} We'll work with ZappTech's shipment data to
answer critical business questions about service levels and
cross-category orders, using the seven mental models of data
manipulation through pandas method chaining.

\begin{tcolorbox}[enhanced jigsaw, breakable, left=2mm, opacitybacktitle=0.6, toptitle=1mm, toprule=.15mm, titlerule=0mm, opacityback=0, coltitle=black, leftrule=.75mm, arc=.35mm, colbacktitle=quarto-callout-warning-color!10!white, rightrule=.15mm, title=\textcolor{quarto-callout-warning-color}{\faExclamationTriangle}\hspace{0.5em}{⚠️ AI Partnership Required}, colback=white, bottomtitle=1mm, bottomrule=.15mm, colframe=quarto-callout-warning-color-frame]

This challenge pushes boundaries intentionally. You'll tackle problems
that normally require weeks of study, but with Cursor AI as your partner
(and your brain keeping it honest), you can accomplish more than you
thought possible.

\textbf{The new reality:} The four stages of competence are Ignorance →
Awareness → Learning → Mastery. AI lets us produce Mastery-level work
while operating primarily in the Awareness stage. I focus on awareness
training, you leverage AI for execution, and together we create outputs
that used to require years of dedicated study.

\end{tcolorbox}

\subsection{The Seven Mental Models of Data
Manipulation}\label{the-seven-mental-models-of-data-manipulation}

The seven most important ways we manipulate datasets are:

\begin{enumerate}
\def\labelenumi{\arabic{enumi}.}
\tightlist
\item
  \textbf{Assign:} Add new variables with calculations and
  transformations
\item
  \textbf{Subset:} Filter data based on conditions or select specific
  columns
\item
  \textbf{Drop:} Remove unwanted variables or observations
\item
  \textbf{Sort:} Arrange data by values or indices
\item
  \textbf{Aggregate:} Summarize data using functions like mean, sum,
  count
\item
  \textbf{Merge:} Combine information from multiple datasets
\item
  \textbf{Split-Apply-Combine:} Group data and apply functions within
  groups
\end{enumerate}

\subsection{Data and Business Context}\label{data-and-business-context}

We analyze ZappTech's shipment data, which contains information about
product deliveries across multiple categories. This dataset is ideal for
our analysis because:

\begin{itemize}
\tightlist
\item
  \textbf{Real Business Questions:} CEO wants to understand service
  levels and cross-category shopping patterns
\item
  \textbf{Multiple Data Sources:} Requires merging shipment data with
  product category information
\item
  \textbf{Complex Relationships:} Service levels may vary by product
  category, and customers may order across categories
\item
  \textbf{Method Chaining Practice:} Perfect for demonstrating all seven
  mental models in sequence
\end{itemize}

\subsection{Data Loading and Initial
Exploration}\label{data-loading-and-initial-exploration}

Let's start by loading the ZappTech shipment data and understanding what
we're working with.

\phantomsection\label{load-data}
\begin{Shaded}
\begin{Highlighting}[]
\ImportTok{import}\NormalTok{ pandas }\ImportTok{as}\NormalTok{ pd}
\ImportTok{import}\NormalTok{ numpy }\ImportTok{as}\NormalTok{ np}
\ImportTok{import}\NormalTok{ matplotlib.pyplot }\ImportTok{as}\NormalTok{ plt}
\ImportTok{import}\NormalTok{ seaborn }\ImportTok{as}\NormalTok{ sns}
\ImportTok{from}\NormalTok{ datetime }\ImportTok{import}\NormalTok{ datetime, timedelta}

\CommentTok{\# Load the shipment data}
\NormalTok{shipments\_df }\OperatorTok{=}\NormalTok{ pd.read\_csv(}
    \StringTok{"https://raw.githubusercontent.com/flyaflya/persuasive/main/shipments.csv"}\NormalTok{, }
\NormalTok{    parse\_dates}\OperatorTok{=}\NormalTok{[}\StringTok{\textquotesingle{}plannedShipDate\textquotesingle{}}\NormalTok{, }\StringTok{\textquotesingle{}actualShipDate\textquotesingle{}}\NormalTok{]}
\NormalTok{)}

\CommentTok{\# Load product line data}
\NormalTok{product\_line\_df }\OperatorTok{=}\NormalTok{ pd.read\_csv(}
    \StringTok{"https://raw.githubusercontent.com/flyaflya/persuasive/main/productLine.csv"}
\NormalTok{)}

\CommentTok{\# Reduce dataset size for faster processing (4,000 rows instead of 96,805 rows)}
\NormalTok{shipments\_df }\OperatorTok{=}\NormalTok{ shipments\_df.head(}\DecValTok{4000}\NormalTok{)}

\BuiltInTok{print}\NormalTok{(}\StringTok{"Shipments data shape:"}\NormalTok{, shipments\_df.shape)}
\BuiltInTok{print}\NormalTok{(}\StringTok{"}\CharTok{\textbackslash{}n}\StringTok{Shipments data columns:"}\NormalTok{, shipments\_df.columns.tolist())}
\BuiltInTok{print}\NormalTok{(}\StringTok{"}\CharTok{\textbackslash{}n}\StringTok{First few rows of shipments data:"}\NormalTok{)}
\BuiltInTok{print}\NormalTok{(shipments\_df.head(}\DecValTok{10}\NormalTok{))}

\BuiltInTok{print}\NormalTok{(}\StringTok{"}\CharTok{\textbackslash{}n}\StringTok{"} \OperatorTok{+} \StringTok{"="}\OperatorTok{*}\DecValTok{50}\NormalTok{)}
\BuiltInTok{print}\NormalTok{(}\StringTok{"Product line data shape:"}\NormalTok{, product\_line\_df.shape)}
\BuiltInTok{print}\NormalTok{(}\StringTok{"}\CharTok{\textbackslash{}n}\StringTok{Product line data columns:"}\NormalTok{, product\_line\_df.columns.tolist())}
\BuiltInTok{print}\NormalTok{(}\StringTok{"}\CharTok{\textbackslash{}n}\StringTok{First few rows of product line data:"}\NormalTok{)}
\BuiltInTok{print}\NormalTok{(product\_line\_df.head(}\DecValTok{10}\NormalTok{))}
\end{Highlighting}
\end{Shaded}

\begin{verbatim}
Shipments data shape: (4000, 5)

Shipments data columns: ['shipID', 'plannedShipDate', 'actualShipDate', 'partID', 'quantity']

First few rows of shipments data:
   shipID plannedShipDate actualShipDate       partID  quantity
0   10001      2013-11-06     2013-10-04  part92b16c5         6
1   10002      2013-10-15     2013-10-04   part66983b         2
2   10003      2013-10-25     2013-10-07  part8e36f25         1
3   10004      2013-10-14     2013-10-08  part30f5de0         1
4   10005      2013-10-14     2013-10-08  part9d64d35         6
5   10006      2013-10-14     2013-10-08  part6cd6167        15
6   10007      2013-10-14     2013-10-08  parta4d5fd1         2
7   10008      2013-10-14     2013-10-08  part08cadf5         1
8   10009      2013-10-14     2013-10-08  part5cc4989        10
9   10010      2013-10-14     2013-10-08  part912ae4c         1

==================================================
Product line data shape: (11997, 3)

Product line data columns: ['partID', 'productLine', 'prodCategory']

First few rows of product line data:
        partID productLine prodCategory
0  part00005ba      line4c      Liquids
1  part000b57d      line61     Machines
2  part00123bf      linec1  Marketables
3  part0021fc9      line61     Machines
4  part0027e86      line2f     Machines
5  part002ed95      line4c      Liquids
6  part0030856      lineb8     Machines
7  part0033dfd      line49      Liquids
8  part0037a2a      linea3  Marketables
9  part003caee      linea3  Marketables
\end{verbatim}

\begin{tcolorbox}[enhanced jigsaw, breakable, left=2mm, opacitybacktitle=0.6, toptitle=1mm, toprule=.15mm, titlerule=0mm, opacityback=0, coltitle=black, leftrule=.75mm, arc=.35mm, colbacktitle=quarto-callout-note-color!10!white, rightrule=.15mm, title=\textcolor{quarto-callout-note-color}{\faInfo}\hspace{0.5em}{💡 Understanding the Data}, colback=white, bottomtitle=1mm, bottomrule=.15mm, colframe=quarto-callout-note-color-frame]

\textbf{Shipments Data:} Contains individual line items for each
shipment, including: - \texttt{shipID}: Unique identifier for each
shipment - \texttt{partID}: Product identifier -
\texttt{plannedShipDate}: When the shipment was supposed to go out -
\texttt{actualShipDate}: When it actually shipped - \texttt{quantity}:
How many units were shipped

\textbf{Product Category and Line Data:} Contains product category
information: - \texttt{partID}: Links to shipments data -
\texttt{productLine}: The category each product belongs to -
\texttt{prodCategory}: The category each product belongs to

\textbf{Business Questions We'll Answer:} 1. Does service level (on-time
shipments) vary across product categories? 2. How often do orders
include products from more than one category?

\end{tcolorbox}

\subsection{The Seven Mental Models: A Progressive Learning
Journey}\label{the-seven-mental-models-a-progressive-learning-journey}

Now we'll work through each of the seven mental models using method
chaining, starting simple and building complexity.

\subsubsection{1. Assign: Adding New
Variables}\label{assign-adding-new-variables}

\textbf{Mental Model:} Create new columns with calculations and
transformations.

Let's start by calculating whether each shipment was late:

\phantomsection\label{mental-model-1-assign}
\begin{Shaded}
\begin{Highlighting}[]
\CommentTok{\# Simple assignment {-} calculate if shipment was late}
\NormalTok{shipments\_with\_lateness }\OperatorTok{=}\NormalTok{ (}
\NormalTok{    shipments\_df}
\NormalTok{    .assign(}
\NormalTok{        is\_late}\OperatorTok{=}\KeywordTok{lambda}\NormalTok{ df: df[}\StringTok{\textquotesingle{}actualShipDate\textquotesingle{}}\NormalTok{] }\OperatorTok{\textgreater{}}\NormalTok{ df[}\StringTok{\textquotesingle{}plannedShipDate\textquotesingle{}}\NormalTok{],}
\NormalTok{        days\_late}\OperatorTok{=}\KeywordTok{lambda}\NormalTok{ df: (df[}\StringTok{\textquotesingle{}actualShipDate\textquotesingle{}}\NormalTok{] }\OperatorTok{{-}}\NormalTok{ df[}\StringTok{\textquotesingle{}plannedShipDate\textquotesingle{}}\NormalTok{]).dt.days}
\NormalTok{    )}
\NormalTok{)}

\BuiltInTok{print}\NormalTok{(}\StringTok{"Added lateness calculations:"}\NormalTok{)}
\BuiltInTok{print}\NormalTok{(shipments\_with\_lateness[[}\StringTok{\textquotesingle{}shipID\textquotesingle{}}\NormalTok{, }\StringTok{\textquotesingle{}plannedShipDate\textquotesingle{}}\NormalTok{, }\StringTok{\textquotesingle{}actualShipDate\textquotesingle{}}\NormalTok{, }\StringTok{\textquotesingle{}is\_late\textquotesingle{}}\NormalTok{, }\StringTok{\textquotesingle{}days\_late\textquotesingle{}}\NormalTok{]].head())}
\end{Highlighting}
\end{Shaded}

\begin{verbatim}
Added lateness calculations:
   shipID plannedShipDate actualShipDate  is_late  days_late
0   10001      2013-11-06     2013-10-04    False        -33
1   10002      2013-10-15     2013-10-04    False        -11
2   10003      2013-10-25     2013-10-07    False        -18
3   10004      2013-10-14     2013-10-08    False         -6
4   10005      2013-10-14     2013-10-08    False         -6
\end{verbatim}

\begin{tcolorbox}[enhanced jigsaw, breakable, left=2mm, opacitybacktitle=0.6, toptitle=1mm, toprule=.15mm, titlerule=0mm, opacityback=0, coltitle=black, leftrule=.75mm, arc=.35mm, colbacktitle=quarto-callout-tip-color!10!white, rightrule=.15mm, title=\textcolor{quarto-callout-tip-color}{\faLightbulb}\hspace{0.5em}{💡 Method Chaining Tip for New Python Users}, colback=white, bottomtitle=1mm, bottomrule=.15mm, colframe=quarto-callout-tip-color-frame]

\textbf{Why use \texttt{lambda\ df:}?} When chaining methods, we need to
reference the current state of the dataframe. The \texttt{lambda\ df:}
tells pandas ``use the current dataframe in this calculation.'' Without
it, pandas would look for a variable called \texttt{df} that doesn't
exist.

\textbf{Alternative approach:} You could also write this as separate
steps, but method chaining keeps related operations together and makes
the code more readable.

\end{tcolorbox}

\begin{tcolorbox}[enhanced jigsaw, breakable, left=2mm, opacitybacktitle=0.6, toptitle=1mm, toprule=.15mm, titlerule=0mm, opacityback=0, coltitle=black, leftrule=.75mm, arc=.35mm, colbacktitle=quarto-callout-important-color!10!white, rightrule=.15mm, title=\textcolor{quarto-callout-important-color}{\faExclamation}\hspace{0.5em}{🤔 Discussion Questions: Assign Mental Model}, colback=white, bottomtitle=1mm, bottomrule=.15mm, colframe=quarto-callout-important-color-frame]

\textbf{Question 1: Data Types and Date Handling} - What is the
\texttt{dtype} of the \texttt{actualShipDate} series? How can you find
out using code? - Why is it important that both \texttt{actualShipDate}
and \texttt{plannedShipDate} have the same data type for comparison?

\textbf{Question 2: String vs Date Comparison} - Can you give an example
where comparing two dates as strings would yield unintuitive results,
e.g.~what happens if you try to compare ``04-11-2025'' and
``05-20-2024'' as strings vs as dates?

\textbf{Question 3: Debug This Code}

\begin{Shaded}
\begin{Highlighting}[]
\CommentTok{\# This code has an error {-} can you spot it?}
\NormalTok{shipments\_with\_lateness }\OperatorTok{=}\NormalTok{ (}
\NormalTok{    shipments\_df}
\NormalTok{    .assign(}
\NormalTok{        is\_late}\OperatorTok{=}\KeywordTok{lambda}\NormalTok{ df: df[}\StringTok{\textquotesingle{}actualShipDate\textquotesingle{}}\NormalTok{] }\OperatorTok{\textgreater{}}\NormalTok{ df[}\StringTok{\textquotesingle{}plannedShipDate\textquotesingle{}}\NormalTok{],}
\NormalTok{        days\_late}\OperatorTok{=}\KeywordTok{lambda}\NormalTok{ df: (df[}\StringTok{\textquotesingle{}actualShipDate\textquotesingle{}}\NormalTok{] }\OperatorTok{{-}}\NormalTok{ df[}\StringTok{\textquotesingle{}plannedShipDate\textquotesingle{}}\NormalTok{]).dt.days,}
\NormalTok{        lateStatement}\OperatorTok{=}\StringTok{"Darn Shipment is Late"} \ControlFlowTok{if}\NormalTok{ shipments\_df[}\StringTok{\textquotesingle{}is\_late\textquotesingle{}}\NormalTok{] }\ControlFlowTok{else} \StringTok{"Shipment is on Time"}
\NormalTok{    )}
\NormalTok{)}
\end{Highlighting}
\end{Shaded}

What's wrong with the \texttt{lateStatement} assignment and how would
you fix it?

\end{tcolorbox}

\paragraph{Briefly Give Answers to the Discussion Questions In This
Section}\label{briefly-give-answers-to-the-discussion-questions-in-this-section}

Adam's answers will be here.

\subsubsection{2. Subset: Querying Rows and Filtering
Columns}\label{subset-querying-rows-and-filtering-columns}

\textbf{Mental Model:} Query rows based on conditions and filter to keep
specific columns.

Let's query for only late shipments and filter to keep the columns we
need:

\phantomsection\label{mental-model-2-subset}
\begin{Shaded}
\begin{Highlighting}[]
\CommentTok{\# Query rows for late shipments and filter to keep specific columns}
\NormalTok{late\_shipments }\OperatorTok{=}\NormalTok{ (}
\NormalTok{    shipments\_with\_lateness}
\NormalTok{    .query(}\StringTok{\textquotesingle{}is\_late == True\textquotesingle{}}\NormalTok{)  }\CommentTok{\# Query rows where is\_late is True}
\NormalTok{    .}\BuiltInTok{filter}\NormalTok{([}\StringTok{\textquotesingle{}shipID\textquotesingle{}}\NormalTok{, }\StringTok{\textquotesingle{}partID\textquotesingle{}}\NormalTok{, }\StringTok{\textquotesingle{}plannedShipDate\textquotesingle{}}\NormalTok{, }\StringTok{\textquotesingle{}actualShipDate\textquotesingle{}}\NormalTok{, }\StringTok{\textquotesingle{}days\_late\textquotesingle{}}\NormalTok{])  }\CommentTok{\# Filter to keep specific columns}
\NormalTok{)}

\BuiltInTok{print}\NormalTok{(}\SpecialStringTok{f"Found }\SpecialCharTok{\{}\BuiltInTok{len}\NormalTok{(late\_shipments)}\SpecialCharTok{\}}\SpecialStringTok{ late shipments out of }\SpecialCharTok{\{}\BuiltInTok{len}\NormalTok{(shipments\_with\_lateness)}\SpecialCharTok{\}}\SpecialStringTok{ total"}\NormalTok{)}
\BuiltInTok{print}\NormalTok{(}\StringTok{"}\CharTok{\textbackslash{}n}\StringTok{Late shipments sample:"}\NormalTok{)}
\BuiltInTok{print}\NormalTok{(late\_shipments.head())}
\end{Highlighting}
\end{Shaded}

\begin{verbatim}
Found 456 late shipments out of 4000 total

Late shipments sample:
     shipID       partID plannedShipDate actualShipDate  days_late
776   10192  part0164a70      2013-10-09     2013-10-14          5
777   10192  part9259836      2013-10-09     2013-10-14          5
778   10192  part4526c73      2013-10-09     2013-10-14          5
779   10192  partbb47e81      2013-10-09     2013-10-14          5
780   10192  part008482f      2013-10-09     2013-10-14          5
\end{verbatim}

\begin{tcolorbox}[enhanced jigsaw, breakable, left=2mm, opacitybacktitle=0.6, toptitle=1mm, toprule=.15mm, titlerule=0mm, opacityback=0, coltitle=black, leftrule=.75mm, arc=.35mm, colbacktitle=quarto-callout-note-color!10!white, rightrule=.15mm, title=\textcolor{quarto-callout-note-color}{\faInfo}\hspace{0.5em}{🔍 Understanding the Methods}, colback=white, bottomtitle=1mm, bottomrule=.15mm, colframe=quarto-callout-note-color-frame]

\begin{itemize}
\tightlist
\item
  \textbf{\texttt{.query()}}: Query rows based on conditions (like SQL
  WHERE clause)
\item
  \textbf{\texttt{.filter()}}: Filter to keep specific columns by name
\item
  \textbf{Alternative}: You could use \texttt{.loc{[}{]}} for more
  complex row querying, but \texttt{.query()} is often more readable
\end{itemize}

\end{tcolorbox}

\begin{tcolorbox}[enhanced jigsaw, breakable, left=2mm, opacitybacktitle=0.6, toptitle=1mm, toprule=.15mm, titlerule=0mm, opacityback=0, coltitle=black, leftrule=.75mm, arc=.35mm, colbacktitle=quarto-callout-important-color!10!white, rightrule=.15mm, title=\textcolor{quarto-callout-important-color}{\faExclamation}\hspace{0.5em}{🤔 Discussion Questions: Subset Mental Model}, colback=white, bottomtitle=1mm, bottomrule=.15mm, colframe=quarto-callout-important-color-frame]

\textbf{Question 1: Query vs Boolean Indexing} - What's the difference
between using
\texttt{.query(\textquotesingle{}is\_late\ ==\ True\textquotesingle{})}
and
\texttt{{[}df{[}\textquotesingle{}is\_late\textquotesingle{}{]}\ ==\ True{]}}?
- Which approach is more readable and why?

\textbf{Question 2: Additional Row Querying} - Can you show an example
of using a variable like \texttt{late\_threshold} to query rows for
shipments that are at least \texttt{late\_threshold} days late,
e.g.~what if you wanted to query rows for shipments that are at least 5
days late?

\end{tcolorbox}

\paragraph{Briefly Give Answers to the Discussion Questions In This
Section}\label{briefly-give-answers-to-the-discussion-questions-in-this-section-1}

\begin{quote}
Replace this with your answers to the discussion questions
\end{quote}

\subsubsection{3. Drop: Removing Unwanted
Data}\label{drop-removing-unwanted-data}

\textbf{Mental Model:} Remove columns or rows you don't need.

Let's clean up our data by removing unnecessary columns:

\phantomsection\label{mental-model-3-drop}
\begin{Shaded}
\begin{Highlighting}[]
\CommentTok{\# Create a cleaner dataset by dropping unnecessary columns}
\NormalTok{clean\_shipments }\OperatorTok{=}\NormalTok{ (}
\NormalTok{    shipments\_with\_lateness}
\NormalTok{    .drop(columns}\OperatorTok{=}\NormalTok{[}\StringTok{\textquotesingle{}quantity\textquotesingle{}}\NormalTok{])  }\CommentTok{\# Drop quantity column (not needed for our analysis)}
\NormalTok{    .dropna(subset}\OperatorTok{=}\NormalTok{[}\StringTok{\textquotesingle{}plannedShipDate\textquotesingle{}}\NormalTok{, }\StringTok{\textquotesingle{}actualShipDate\textquotesingle{}}\NormalTok{])  }\CommentTok{\# Remove rows with missing dates}
\NormalTok{)}

\BuiltInTok{print}\NormalTok{(}\SpecialStringTok{f"Cleaned dataset: }\SpecialCharTok{\{}\BuiltInTok{len}\NormalTok{(clean\_shipments)}\SpecialCharTok{\}}\SpecialStringTok{ rows, }\SpecialCharTok{\{}\BuiltInTok{len}\NormalTok{(clean\_shipments.columns)}\SpecialCharTok{\}}\SpecialStringTok{ columns"}\NormalTok{)}
\BuiltInTok{print}\NormalTok{(}\StringTok{"Remaining columns:"}\NormalTok{, clean\_shipments.columns.tolist())}
\end{Highlighting}
\end{Shaded}

\begin{verbatim}
Cleaned dataset: 4000 rows, 6 columns
Remaining columns: ['shipID', 'plannedShipDate', 'actualShipDate', 'partID', 'is_late', 'days_late']
\end{verbatim}

\begin{tcolorbox}[enhanced jigsaw, breakable, left=2mm, opacitybacktitle=0.6, toptitle=1mm, toprule=.15mm, titlerule=0mm, opacityback=0, coltitle=black, leftrule=.75mm, arc=.35mm, colbacktitle=quarto-callout-important-color!10!white, rightrule=.15mm, title=\textcolor{quarto-callout-important-color}{\faExclamation}\hspace{0.5em}{🤔 Discussion Questions: Drop Mental Model}, colback=white, bottomtitle=1mm, bottomrule=.15mm, colframe=quarto-callout-important-color-frame]

\textbf{Question 1: Drop vs Filter Strategies} - What's the difference
between
\texttt{.drop(columns={[}\textquotesingle{}quantity\textquotesingle{}{]})}
and \texttt{.filter()} with a list of columns you want to keep? - When
would you choose to drop columns vs filter to keep specific columns?

\textbf{Question 2: Handling Missing Data} - What happens if you use
\texttt{.dropna()} without specifying \texttt{subset}? How is this
different from
\texttt{.dropna(subset={[}\textquotesingle{}plannedShipDate\textquotesingle{},\ \textquotesingle{}actualShipDate\textquotesingle{}{]})}?
- Why might you want to be selective about which columns to check for
missing values?

\end{tcolorbox}

\paragraph{Briefly Give Answers to the Discussion Questions In This
Section}\label{briefly-give-answers-to-the-discussion-questions-in-this-section-2}

\begin{quote}
Replace this with your answers to the discussion questions
\end{quote}

\subsubsection{4. Sort: Arranging Data}\label{sort-arranging-data}

\textbf{Mental Model:} Order data by values or indices.

Let's sort by lateness to see the worst offenders:

\phantomsection\label{mental-model-4-sort}
\begin{Shaded}
\begin{Highlighting}[]
\CommentTok{\# Sort by days late (worst first)}
\NormalTok{sorted\_by\_lateness }\OperatorTok{=}\NormalTok{ (}
\NormalTok{    clean\_shipments}
\NormalTok{    .sort\_values(}\StringTok{\textquotesingle{}days\_late\textquotesingle{}}\NormalTok{, ascending}\OperatorTok{=}\VariableTok{False}\NormalTok{)  }\CommentTok{\# Sort by days\_late, highest first}
\NormalTok{    .reset\_index(drop}\OperatorTok{=}\VariableTok{True}\NormalTok{)  }\CommentTok{\# Reset index to be sequential}
\NormalTok{)}

\BuiltInTok{print}\NormalTok{(}\StringTok{"Shipments sorted by lateness (worst first):"}\NormalTok{)}
\BuiltInTok{print}\NormalTok{(sorted\_by\_lateness[[}\StringTok{\textquotesingle{}shipID\textquotesingle{}}\NormalTok{, }\StringTok{\textquotesingle{}partID\textquotesingle{}}\NormalTok{, }\StringTok{\textquotesingle{}days\_late\textquotesingle{}}\NormalTok{, }\StringTok{\textquotesingle{}is\_late\textquotesingle{}}\NormalTok{]].head(}\DecValTok{10}\NormalTok{))}
\end{Highlighting}
\end{Shaded}

\begin{verbatim}
Shipments sorted by lateness (worst first):
   shipID       partID  days_late  is_late
0   10956  part795d1a4         21     True
1   10956  partf23fd1e         21     True
2   10956  partc653823         21     True
3   10956  partb6208b5         21     True
4   10956  parte820e31         21     True
5   10956  part50c6b9a         21     True
6   10956  part1fedfcf         21     True
7   10956  part3017fa1         21     True
8   10956  part66bb851         21     True
9   10956  partd5b19e4         21     True
\end{verbatim}

\begin{tcolorbox}[enhanced jigsaw, breakable, left=2mm, opacitybacktitle=0.6, toptitle=1mm, toprule=.15mm, titlerule=0mm, opacityback=0, coltitle=black, leftrule=.75mm, arc=.35mm, colbacktitle=quarto-callout-important-color!10!white, rightrule=.15mm, title=\textcolor{quarto-callout-important-color}{\faExclamation}\hspace{0.5em}{🤔 Discussion Questions: Sort Mental Model}, colback=white, bottomtitle=1mm, bottomrule=.15mm, colframe=quarto-callout-important-color-frame]

\textbf{Question 1: Sorting Strategies} - What's the difference between
\texttt{ascending=False} and \texttt{ascending=True} in sorting? - How
would you sort by multiple columns (e.g., first by \texttt{is\_late},
then by \texttt{days\_late})?

\textbf{Question 2: Index Management} - Why do we use
\texttt{.reset\_index(drop=True)} after sorting? - What happens to the
original index when you sort? Why might this be problematic?

\end{tcolorbox}

\paragraph{Briefly Give Answers to the Discussion Questions In This
Section}\label{briefly-give-answers-to-the-discussion-questions-in-this-section-3}

\begin{quote}
Replace this with your answers to the discussion questions
\end{quote}

\subsubsection{5. Aggregate: Summarizing
Data}\label{aggregate-summarizing-data}

\textbf{Mental Model:} Calculate summary statistics across groups or the
entire dataset.

Let's calculate overall service level metrics:

\phantomsection\label{mental-model-5-aggregate}
\begin{Shaded}
\begin{Highlighting}[]
\CommentTok{\# Calculate overall service level metrics}
\NormalTok{service\_metrics }\OperatorTok{=}\NormalTok{ (}
\NormalTok{    clean\_shipments}
\NormalTok{    .agg(\{}
        \StringTok{\textquotesingle{}is\_late\textquotesingle{}}\NormalTok{: [}\StringTok{\textquotesingle{}count\textquotesingle{}}\NormalTok{, }\StringTok{\textquotesingle{}sum\textquotesingle{}}\NormalTok{, }\StringTok{\textquotesingle{}mean\textquotesingle{}}\NormalTok{],  }\CommentTok{\# Count total, count late, calculate percentage}
        \StringTok{\textquotesingle{}days\_late\textquotesingle{}}\NormalTok{: [}\StringTok{\textquotesingle{}mean\textquotesingle{}}\NormalTok{, }\StringTok{\textquotesingle{}max\textquotesingle{}}\NormalTok{]  }\CommentTok{\# Average and maximum days late}
\NormalTok{    \})}
\NormalTok{    .}\BuiltInTok{round}\NormalTok{(}\DecValTok{3}\NormalTok{)}
\NormalTok{)}

\BuiltInTok{print}\NormalTok{(}\StringTok{"Overall Service Level Metrics:"}\NormalTok{)}
\BuiltInTok{print}\NormalTok{(service\_metrics)}

\CommentTok{\# Calculate percentage on{-}time directly from the data}
\NormalTok{on\_time\_rate }\OperatorTok{=}\NormalTok{ (}\DecValTok{1} \OperatorTok{{-}}\NormalTok{ clean\_shipments[}\StringTok{\textquotesingle{}is\_late\textquotesingle{}}\NormalTok{].mean()) }\OperatorTok{*} \DecValTok{100}
\BuiltInTok{print}\NormalTok{(}\SpecialStringTok{f"}\CharTok{\textbackslash{}n}\SpecialStringTok{On{-}time delivery rate: }\SpecialCharTok{\{}\NormalTok{on\_time\_rate}\SpecialCharTok{:.1f\}}\SpecialStringTok{\%"}\NormalTok{)}
\end{Highlighting}
\end{Shaded}

\begin{verbatim}
Overall Service Level Metrics:
        is_late  days_late
count  4000.000        NaN
sum     456.000        NaN
mean      0.114     -0.974
max         NaN     21.000

On-time delivery rate: 88.6%
\end{verbatim}

\begin{tcolorbox}[enhanced jigsaw, breakable, left=2mm, opacitybacktitle=0.6, toptitle=1mm, toprule=.15mm, titlerule=0mm, opacityback=0, coltitle=black, leftrule=.75mm, arc=.35mm, colbacktitle=quarto-callout-important-color!10!white, rightrule=.15mm, title=\textcolor{quarto-callout-important-color}{\faExclamation}\hspace{0.5em}{🤔 Discussion Questions: Aggregate Mental Model}, colback=white, bottomtitle=1mm, bottomrule=.15mm, colframe=quarto-callout-important-color-frame]

\textbf{Question 1: Boolean Aggregation} - Why does \texttt{sum()} work
on boolean values? What does it count?

\end{tcolorbox}

\paragraph{Briefly Give Answers to the Discussion Questions In This
Section}\label{briefly-give-answers-to-the-discussion-questions-in-this-section-4}

\begin{quote}
Replace this with your answers to the discussion questions
\end{quote}

\subsubsection{6. Merge: Combining
Information}\label{merge-combining-information}

\textbf{Mental Model:} Join data from multiple sources to create richer
datasets.

Now let's analyze service levels by product category. First, we need to
merge our data:

\phantomsection\label{mental-model-6-merge-prep}
\begin{Shaded}
\begin{Highlighting}[]
\CommentTok{\# Merge shipment data with product line data}
\NormalTok{shipments\_with\_category }\OperatorTok{=}\NormalTok{ (}
\NormalTok{    clean\_shipments}
\NormalTok{    .merge(product\_line\_df, on}\OperatorTok{=}\StringTok{\textquotesingle{}partID\textquotesingle{}}\NormalTok{, how}\OperatorTok{=}\StringTok{\textquotesingle{}left\textquotesingle{}}\NormalTok{)  }\CommentTok{\# Left join to keep all shipments}
\NormalTok{    .assign(}
\NormalTok{        category\_late}\OperatorTok{=}\KeywordTok{lambda}\NormalTok{ df: df[}\StringTok{\textquotesingle{}is\_late\textquotesingle{}}\NormalTok{] }\OperatorTok{\&}\NormalTok{ df[}\StringTok{\textquotesingle{}prodCategory\textquotesingle{}}\NormalTok{].notna()  }\CommentTok{\# Only count as late if we have category info}
\NormalTok{    )}
\NormalTok{)}

\BuiltInTok{print}\NormalTok{(}\StringTok{"}\CharTok{\textbackslash{}n}\StringTok{Product categories available:"}\NormalTok{)}
\BuiltInTok{print}\NormalTok{(shipments\_with\_category[}\StringTok{\textquotesingle{}prodCategory\textquotesingle{}}\NormalTok{].value\_counts())}
\end{Highlighting}
\end{Shaded}

\begin{verbatim}

Product categories available:
prodCategory
Marketables    1850
Machines        846
SpareParts      767
Liquids         537
Name: count, dtype: int64
\end{verbatim}

\begin{tcolorbox}[enhanced jigsaw, breakable, left=2mm, opacitybacktitle=0.6, toptitle=1mm, toprule=.15mm, titlerule=0mm, opacityback=0, coltitle=black, leftrule=.75mm, arc=.35mm, colbacktitle=quarto-callout-important-color!10!white, rightrule=.15mm, title=\textcolor{quarto-callout-important-color}{\faExclamation}\hspace{0.5em}{🤔 Discussion Questions: Merge Mental Model}, colback=white, bottomtitle=1mm, bottomrule=.15mm, colframe=quarto-callout-important-color-frame]

\textbf{Question 1: Join Types and Data Loss} - Why does your professor
think we should use
\texttt{how=\textquotesingle{}left\textquotesingle{}} in most cases? -
How can you check if any shipments were lost during the merge?

\textbf{Question 2: Key Column Matching} - What happens if there are
duplicate \texttt{partID} values in the \texttt{product\_line\_df}?

\end{tcolorbox}

\paragraph{Briefly Give Answers to the Discussion Questions In This
Section}\label{briefly-give-answers-to-the-discussion-questions-in-this-section-5}

\begin{quote}
Replace this with your answers to the discussion questions
\end{quote}

\subsubsection{7. Split-Apply-Combine: Group
Analysis}\label{split-apply-combine-group-analysis}

\textbf{Mental Model:} Group data and apply functions within each group.

Now let's analyze service levels by category:

\phantomsection\label{mental-model-7-groupby}
\begin{Shaded}
\begin{Highlighting}[]
\CommentTok{\# Analyze service levels by product category}
\NormalTok{service\_by\_category }\OperatorTok{=}\NormalTok{ (}
\NormalTok{    shipments\_with\_category}
\NormalTok{    .groupby(}\StringTok{\textquotesingle{}prodCategory\textquotesingle{}}\NormalTok{)  }\CommentTok{\# Split by product category}
\NormalTok{    .agg(\{}
        \StringTok{\textquotesingle{}is\_late\textquotesingle{}}\NormalTok{: [}\StringTok{\textquotesingle{}any\textquotesingle{}}\NormalTok{, }\StringTok{\textquotesingle{}count\textquotesingle{}}\NormalTok{, }\StringTok{\textquotesingle{}sum\textquotesingle{}}\NormalTok{, }\StringTok{\textquotesingle{}mean\textquotesingle{}}\NormalTok{],  }\CommentTok{\# Count, late count, percentage late}
        \StringTok{\textquotesingle{}days\_late\textquotesingle{}}\NormalTok{: [}\StringTok{\textquotesingle{}mean\textquotesingle{}}\NormalTok{, }\StringTok{\textquotesingle{}max\textquotesingle{}}\NormalTok{]  }\CommentTok{\# Average and max days late}
\NormalTok{    \})}
\NormalTok{    .}\BuiltInTok{round}\NormalTok{(}\DecValTok{3}\NormalTok{)}
\NormalTok{)}

\BuiltInTok{print}\NormalTok{(}\StringTok{"Service Level by Product Category:"}\NormalTok{)}
\BuiltInTok{print}\NormalTok{(service\_by\_category)}
\end{Highlighting}
\end{Shaded}

\begin{verbatim}
Service Level by Product Category:
             is_late                   days_late    
                 any count  sum   mean      mean max
prodCategory                                        
Liquids         True   537   22  0.041    -0.950  19
Machines        True   846  152  0.180    -1.336  21
Marketables     True  1850  145  0.078    -0.804  21
SpareParts      True   767  137  0.179    -1.003  21
\end{verbatim}

\begin{tcolorbox}[enhanced jigsaw, breakable, left=2mm, opacitybacktitle=0.6, toptitle=1mm, toprule=.15mm, titlerule=0mm, opacityback=0, coltitle=black, leftrule=.75mm, arc=.35mm, colbacktitle=quarto-callout-important-color!10!white, rightrule=.15mm, title=\textcolor{quarto-callout-important-color}{\faExclamation}\hspace{0.5em}{🤔 Discussion Questions: Split-Apply-Combine Mental Model}, colback=white, bottomtitle=1mm, bottomrule=.15mm, colframe=quarto-callout-important-color-frame]

\textbf{Question 1: GroupBy Mechanics} - What does
\texttt{.groupby(\textquotesingle{}prodCategory\textquotesingle{})}
actually do? How does it ``split'' the data? - Why do we need to use
\texttt{.agg()} after grouping? What happens if you don't?

\textbf{Question 2: Multi-Level Grouping} - Explore grouping by
\texttt{{[}\textquotesingle{}shipID\textquotesingle{},\ \textquotesingle{}prodCategory\textquotesingle{}{]}}?
What question does this answer versus grouping by
\texttt{\textquotesingle{}prodCategory\textquotesingle{}} alone? (HINT:
There may be many rows with identical shipID's due to a particular order
having multiple partID's.)

\end{tcolorbox}

\paragraph{Briefly Give Answers to the Discussion Questions In This
Section}\label{briefly-give-answers-to-the-discussion-questions-in-this-section-6}

\begin{quote}
Replace this with your answers to the discussion questions
\end{quote}

\subsection{Answering A Business
Question}\label{answering-a-business-question}

\textbf{Mental Model:} Combine multiple data manipulation techniques to
answer complex business questions.

Let's create a comprehensive analysis by combining shipment-level data
with category information:

\phantomsection\label{mental-model-7-comprehensive}
\begin{Shaded}
\begin{Highlighting}[]
\CommentTok{\# Create a comprehensive analysis dataset}
\NormalTok{comprehensive\_analysis }\OperatorTok{=}\NormalTok{ (}
\NormalTok{    shipments\_with\_category}
\NormalTok{    .groupby([}\StringTok{\textquotesingle{}shipID\textquotesingle{}}\NormalTok{, }\StringTok{\textquotesingle{}prodCategory\textquotesingle{}}\NormalTok{])  }\CommentTok{\# Group by shipment and category}
\NormalTok{    .agg(\{}
        \StringTok{\textquotesingle{}is\_late\textquotesingle{}}\NormalTok{: }\StringTok{\textquotesingle{}any\textquotesingle{}}\NormalTok{,  }\CommentTok{\# True if any item in this shipment/category is late}
        \StringTok{\textquotesingle{}days\_late\textquotesingle{}}\NormalTok{: }\StringTok{\textquotesingle{}max\textquotesingle{}}  \CommentTok{\# Maximum days late for this shipment/category}
\NormalTok{    \})}
\NormalTok{    .reset\_index()}
\NormalTok{    .assign(}
\NormalTok{        has\_multiple\_categories}\OperatorTok{=}\KeywordTok{lambda}\NormalTok{ df: df.groupby(}\StringTok{\textquotesingle{}shipID\textquotesingle{}}\NormalTok{)[}\StringTok{\textquotesingle{}prodCategory\textquotesingle{}}\NormalTok{].transform(}\StringTok{\textquotesingle{}nunique\textquotesingle{}}\NormalTok{) }\OperatorTok{\textgreater{}} \DecValTok{1}
\NormalTok{    )}
\NormalTok{)}

\BuiltInTok{print}\NormalTok{(}\StringTok{"Comprehensive analysis {-} shipments with multiple categories:"}\NormalTok{)}
\NormalTok{multi\_category\_shipments }\OperatorTok{=}\NormalTok{ comprehensive\_analysis[comprehensive\_analysis[}\StringTok{\textquotesingle{}has\_multiple\_categories\textquotesingle{}}\NormalTok{]]}
\BuiltInTok{print}\NormalTok{(}\SpecialStringTok{f"Shipments with multiple categories: }\SpecialCharTok{\{}\NormalTok{multi\_category\_shipments[}\StringTok{\textquotesingle{}shipID\textquotesingle{}}\NormalTok{]}\SpecialCharTok{.}\NormalTok{nunique()}\SpecialCharTok{\}}\SpecialStringTok{"}\NormalTok{)}
\BuiltInTok{print}\NormalTok{(}\SpecialStringTok{f"Total unique shipments: }\SpecialCharTok{\{}\NormalTok{comprehensive\_analysis[}\StringTok{\textquotesingle{}shipID\textquotesingle{}}\NormalTok{]}\SpecialCharTok{.}\NormalTok{nunique()}\SpecialCharTok{\}}\SpecialStringTok{"}\NormalTok{)}
\BuiltInTok{print}\NormalTok{(}\SpecialStringTok{f"Percentage with multiple categories: }\SpecialCharTok{\{}\NormalTok{multi\_category\_shipments[}\StringTok{\textquotesingle{}shipID\textquotesingle{}}\NormalTok{]}\SpecialCharTok{.}\NormalTok{nunique() }\OperatorTok{/}\NormalTok{ comprehensive\_analysis[}\StringTok{\textquotesingle{}shipID\textquotesingle{}}\NormalTok{]}\SpecialCharTok{.}\NormalTok{nunique() }\OperatorTok{*} \DecValTok{100}\SpecialCharTok{:.1f\}}\SpecialStringTok{\%"}\NormalTok{)}
\end{Highlighting}
\end{Shaded}

\begin{verbatim}
Comprehensive analysis - shipments with multiple categories:
Shipments with multiple categories: 232
Total unique shipments: 997
Percentage with multiple categories: 23.3%
\end{verbatim}

\begin{tcolorbox}[enhanced jigsaw, breakable, left=2mm, opacitybacktitle=0.6, toptitle=1mm, toprule=.15mm, titlerule=0mm, opacityback=0, coltitle=black, leftrule=.75mm, arc=.35mm, colbacktitle=quarto-callout-important-color!10!white, rightrule=.15mm, title=\textcolor{quarto-callout-important-color}{\faExclamation}\hspace{0.5em}{🤔 Discussion Questions: Answering A Business Question}, colback=white, bottomtitle=1mm, bottomrule=.15mm, colframe=quarto-callout-important-color-frame]

\textbf{Question 1: Business Question Analysis} - What business question
does this comprehensive analysis answer? - How does grouping by
\texttt{{[}\textquotesingle{}shipID\textquotesingle{},\ \textquotesingle{}prodCategory\textquotesingle{}{]}}
differ from grouping by just
\texttt{\textquotesingle{}prodCategory\textquotesingle{}}? - What
insights can ZappTech's management gain from knowing the percentage of
multi-category shipments?

\end{tcolorbox}

\paragraph{Briefly Give Answers to the Discussion Questions In This
Section}\label{briefly-give-answers-to-the-discussion-questions-in-this-section-7}

\begin{quote}
Replace this with your answers to the discussion questions
\end{quote}

\subsection{Student Analysis Section: Mastering Data
Manipulation}\label{student-analysis-section}

\textbf{Your Task:} Demonstrate your mastery of the seven mental models
through comprehensive discussion and analysis. The bulk of your grade
comes from thoughtfully answering the discussion questions for each
mental model. See below for more details.

\subsubsection{Core Challenge: Discussion Questions
Analysis}\label{core-challenge-discussion-questions-analysis}

\textbf{For each mental model, provide:} - Clear, concise answers to all
discussion questions - Code examples where appropriate to support your
explanations

\begin{tcolorbox}[enhanced jigsaw, breakable, left=2mm, opacitybacktitle=0.6, toptitle=1mm, toprule=.15mm, titlerule=0mm, opacityback=0, coltitle=black, leftrule=.75mm, arc=.35mm, colbacktitle=quarto-callout-important-color!10!white, rightrule=.15mm, title=\textcolor{quarto-callout-important-color}{\faExclamation}\hspace{0.5em}{📊 Discussion Questions Requirements}, colback=white, bottomtitle=1mm, bottomrule=.15mm, colframe=quarto-callout-important-color-frame]

\textbf{Complete all discussion question sections:} 1. \textbf{Assign
Mental Model:} Data types, date handling, and debugging 2.
\textbf{Subset Mental Model:} Filtering strategies and complex queries
3. \textbf{Drop Mental Model:} Data cleaning and quality management 4.
\textbf{Sort Mental Model:} Data organization and business logic 5.
\textbf{Aggregate Mental Model:} Summary statistics and business metrics
6. \textbf{Merge Mental Model:} Data integration and quality control 7.
\textbf{Split-Apply-Combine Mental Model:} Group analysis and advanced
operations 8. \textbf{Answering A Business Question:} Combining multiple
data manipulation techniques to answer a business question

\end{tcolorbox}

\subsubsection{Professional Visualizations (For 100\%
Grade)}\label{professional-visualizations-for-100-grade}

\textbf{Your Task:} Create a professional visualization that supports
your analysis and demonstrates your understanding of the data.

\textbf{Create visualizations showing:} - Service level (on-time
percentage) by product category

\textbf{Your visualizations should:} - Use clear labels and professional
formatting - Support the insights from your discussion questions - Be
appropriate for a business audience - Do not \texttt{echo} the code that
creates the visualizations

\subsection{Challenge Requirements 📋}\label{challenge-requirements}

\textbf{Your Primary Task:} Answer all discussion questions for the
seven mental models with thoughtful, well-reasoned responses that
demonstrate your understanding of data manipulation concepts.

\textbf{Key Requirements:} - Complete discussion questions for each
mental model - Demonstrate clear understanding of pandas concepts and
data manipulation techniques - Write clear, business-focused analysis
that explains your findings

\subsection{Getting Started: Repository Setup
🚀}\label{getting-started-repository-setup}

\begin{tcolorbox}[enhanced jigsaw, breakable, left=2mm, opacitybacktitle=0.6, toptitle=1mm, toprule=.15mm, titlerule=0mm, opacityback=0, coltitle=black, leftrule=.75mm, arc=.35mm, colbacktitle=quarto-callout-important-color!10!white, rightrule=.15mm, title=\textcolor{quarto-callout-important-color}{\faExclamation}\hspace{0.5em}{📁 Getting Started}, colback=white, bottomtitle=1mm, bottomrule=.15mm, colframe=quarto-callout-important-color-frame]

\textbf{Step 1:} Fork and clone this challenge repository - Go to the
course repository and find the ``dataManipulationChallenge'' folder -
Fork it to your GitHub account, or clone it directly - Open the cloned
repository in Cursor

\textbf{Step 2:} Set up your Python environment - Follow the Python
setup instructions above (use your existing venv from Tech Setup
Challenge Part 2) - Make sure your virtual environment is activated and
the Python interpreter is set

\textbf{Step 3:} You're ready to start! The data loading code is already
provided in this file.

\textbf{Note:} This challenge uses the same \texttt{index.qmd} file
you're reading right now - you'll edit it to complete your analysis.

\end{tcolorbox}

\subsubsection{Getting Started Tips}\label{getting-started-tips}

\begin{tcolorbox}[enhanced jigsaw, breakable, left=2mm, opacitybacktitle=0.6, toptitle=1mm, toprule=.15mm, titlerule=0mm, opacityback=0, coltitle=black, leftrule=.75mm, arc=.35mm, colbacktitle=quarto-callout-note-color!10!white, rightrule=.15mm, title=\textcolor{quarto-callout-note-color}{\faInfo}\hspace{0.5em}{🎯 Method Chaining Philosophy}, colback=white, bottomtitle=1mm, bottomrule=.15mm, colframe=quarto-callout-note-color-frame]

\begin{quote}
``Each operation should build naturally on the previous one''
\end{quote}

\emph{Think of method chaining like building with LEGO blocks - each
piece connects to the next, creating something more complex and useful
than the individual pieces.}

\end{tcolorbox}

\begin{tcolorbox}[enhanced jigsaw, breakable, left=2mm, opacitybacktitle=0.6, toptitle=1mm, toprule=.15mm, titlerule=0mm, opacityback=0, coltitle=black, leftrule=.75mm, arc=.35mm, colbacktitle=quarto-callout-warning-color!10!white, rightrule=.15mm, title=\textcolor{quarto-callout-warning-color}{\faExclamationTriangle}\hspace{0.5em}{💾 Important: Save Your Work Frequently!}, colback=white, bottomtitle=1mm, bottomrule=.15mm, colframe=quarto-callout-warning-color-frame]

\textbf{Before you start:} Make sure to commit your work often using the
Source Control panel in Cursor (Ctrl+Shift+G or Cmd+Shift+G). This
prevents the AI from overwriting your progress and ensures you don't
lose your work.

\textbf{Commit after each major step:}

\begin{itemize}
\tightlist
\item
  After completing each mental model section
\item
  After adding your visualizations
\item
  After completing your advanced method chain
\item
  Before asking the AI for help with new code
\end{itemize}

\textbf{How to commit:}

\begin{enumerate}
\def\labelenumi{\arabic{enumi}.}
\tightlist
\item
  Open Source Control panel (Ctrl+Shift+G)
\item
  Stage your changes (+ button)
\item
  Write a descriptive commit message
\item
  Click the checkmark to commit
\end{enumerate}

\emph{Remember: Frequent commits are your safety net!}

\end{tcolorbox}

\subsection{Grading Rubric 🎓}\label{grading-rubric}

\textbf{75\% Grade:} Complete discussion questions for at least 5 of the
7 mental models with clear, thoughtful responses.

\textbf{85\% Grade:} Complete discussion questions for all 7 mental
models with comprehensive, well-reasoned responses.

\textbf{95\% Grade:} Complete all discussion questions plus the
``Answering A Business Question'' section.

\textbf{100\% Grade:} Complete all discussion questions plus create a
professional visualization showing service level by product category.

\subsection{Submission Checklist ✅}\label{submission-checklist}

\textbf{Minimum Requirements (Required for Any Points):}

\begin{itemize}
\tightlist
\item[$\square$]
  Created repository named ``dataManipulationChallenge'' in your GitHub
  account
\item[$\square$]
  Cloned repository locally using Cursor (or VS Code)
\item[$\square$]
  Completed discussion questions for at least 5 of the 7 mental models
\item[$\square$]
  Document rendered to HTML successfully
\item[$\square$]
  HTML files uploaded to your repository
\item[$\square$]
  GitHub Pages enabled and working
\item[$\square$]
  Site accessible at
  \texttt{https://{[}your-username{]}.github.io/dataManipulationChallenge/}
\end{itemize}

\textbf{75\% Grade Requirements:}

\begin{itemize}
\tightlist
\item[$\square$]
  Complete discussion questions for at least 5 of the 7 mental models
\item[$\square$]
  Clear, thoughtful responses that demonstrate understanding
\item[$\square$]
  Code examples where appropriate to support explanations
\end{itemize}

\textbf{85\% Grade Requirements:}

\begin{itemize}
\tightlist
\item[$\square$]
  Complete discussion questions for all 7 mental models
\item[$\square$]
  Comprehensive, well-reasoned responses showing deep understanding
\item[$\square$]
  Business context for why concepts matter
\item[$\square$]
  Examples of real-world applications
\end{itemize}

\textbf{95\% Grade Requirements:}

\begin{itemize}
\tightlist
\item[$\square$]
  Complete discussion questions for all 7 mental models
\item[$\square$]
  Complete the ``Answering A Business Question'' discussion questions
\item[$\square$]
  Comprehensive, well-reasoned responses showing deep understanding
\item[$\square$]
  Business context for why concepts matter
\end{itemize}

\textbf{100\% Grade Requirements:}

\begin{itemize}
\tightlist
\item[$\square$]
  All discussion questions completed with professional quality
\item[$\square$]
  Professional visualization showing service level by product category
\item[$\square$]
  Professional presentation style appropriate for business audience
\item[$\square$]
  Clear, engaging narrative that tells a compelling story
\item[$\square$]
  Practical insights that would help ZappTech's management
\end{itemize}

\textbf{Report Quality (Critical for Higher Grades):}

\begin{itemize}
\tightlist
\item[$\square$]
  Professional writing style (no AI-generated fluff)
\item[$\square$]
  Concise analysis that gets to the point
\end{itemize}




\end{document}
